% TeX 
%
% Latex template for MIUN Licentiate/PhD thesis
% Author: David Krapohl
% Author: Winnie Wong
\input glyphtounicode                                               % use unicode for pdf
 \pdfgentounicode=1

% original 10pt
\documentclass[11pt,twoside,openright]{memoir}
%showtrims
\usepackage{fixltx2e}							%this package fixes some bugs in latex in case you have problems. I needed to use it in my thesis but don't remember for what

\usepackage[T1]{fontenc}
\usepackage[utf8]{inputenc}
\usepackage{textcomp}

\usepackage{etoolbox}

%\usepackage[ibycus,english,swedish,german]{babel}
\usepackage[english,swedish]{babel}

\usepackage{csquotes}
\usepackage{nicefrac}

% -- Graphics & Fonts & Colors
\usepackage{amssymb}
\usepackage{newpxtext,newpxmath}			% palatino (MIUN,serif)
%\usepackage{amsmath, pxfonts}				% palatino math
%\usepackage{gillius}						% gill sans (MIUN, sans)
\usepackage{helvet}							% helvetica as sans font

\usepackage{textgreek}

\usepackage[kerning=true,
            tracking=true,
            spacing=true,
            factor=1100,
            activate={true,nocompatibility},
            final]{microtype}				% even grayvalue, less rivers, optical edges
\SetTracking{encoding={*}, shape=sc}{40}	% nicer SC shape, less dense
\linespread{1.05}

\usepackage[toc,eqno]{tabfigures}	% set lining figures in tables

\usepackage[usenames,dvipsnames,cmyk]{xcolor}
    % define MIUN colors
    \definecolor{hiddenLink}{rgb}{0,0,0}
    \definecolor{miunBlue}{cmyk}{1,0.34,0,0.2}
    \definecolor{miunYellow}{cmyk}{0,0.1,1,0}
    \definecolor{miunBlack}{cmyk}{0,0,0,1}
    \definecolor{miunLightGray}{cmyk}{0,0,0,0.11}
    \definecolor{miunDarkGray}{cmyk}{0.11,0.11,0,0.69}

\usepackage[pdftex]{graphicx}
    \graphicspath{{./figures/}}

% useful extensions
\usepackage{makeidx}				

\usepackage{caption}                       % configure captions
%\usepackage{subcaption}
\usepackage{wrapfig}                        % wraps figure with text
\usepackage{float} 			% allows floating objects
%\usepackage[printonlyused]{acronym}        % print list of acronyms
\usepackage{booktabs}                       % beautify tables
\usepackage{multirow}                       % multiple rows in tables
\usepackage{siunitx}                        % make setting units fun and good looking 
\usepackage[version=3]{mhchem}              % nicer isotopes

\AtBeginEnvironment{tabular}{%
    \figureversion{lf,tab}
    \sisetup{text-rm={\figureversion{tab,lf}}}
}
    \sisetup{detect-weight=true, detect-family=true, detect-mode=true, tight-spacing=true} % Make siunitx detect font-face and weight


\usepackage[caption=false,font=footnotesize]{subfig}
    %\renewcommand{\arraystretch}{1}				  % more space in tables
%\usepackage{rotating}
\usepackage{longtable}
\usepackage{tabu}
\usepackage{gensymb}

\usepackage{enumitem}
% redefine the description environment
    \setlist[description]{%
        %    topsep=30pt,               	% space before start / after end of list
        %    itemsep=5pt,               	% space between items
        font={\normalfont\scshape}, % set the label font
        labelsep=\textwidth
    }

\usepackage{pdfpages}						% helps including pdf
\usepackage{tikz}							% graph drawing
\usepackage{pgfplots}						% plot in tikz
\pgfplotsset{compat=1.14}
% !Tex root = ../main.tex

\pgfplotsset{compat=1.10}
\usepgfplotslibrary{fillbetween}

\usetikzlibrary{arrows,positioning,snakes,shapes} 

% some tikz libraries
\usetikzlibrary{decorations.pathmorphing, decorations.pathreplacing, patterns}
\usetikzlibrary{circuits.ee.IEC}
\usetikzlibrary{calc}
% externalize all tikz figures: faster compilations
% pdf, log and md5sum is in tikz folder
\usetikzlibrary{external}
% externalize does not like includepdf
\tikzexternalize[prefix=tikz/, shell escape=-enable-write18, optimize command away=\includepdf]

\tikzset{
    %Define standard arrow tip
    >=stealth',
%    %Define style for boxes
%    point/.style={
%           rectangle,
%           rounded corners,
%           draw=black, very thick,
%           text width=6.5em,
%           minimum height=2em,
%           text centere},
	font=\sffamily,
    % Define arrow style
    arrow/.style={
           ->,
           thick,
           shorten <=2pt,
           shorten >=2pt,}
%    external/optimize=false
}
%some commands for figures that are reused
\newcommand{\electron}[1]{%
	\shade[ball color=darkgray] (#1) circle (.1);\draw (#1) node{};
}

\newcommand{\hole}[1]{%
	\shade[ball color=white] (#1) circle (.1);
}
%\tikzifexternalizing{% compile pgf only when it changes
%% don't include package XYZ here
%}{%
%\usepackage{pdfpages}% helps including pdf
%}%
\usepackage{hyperref}						% links in pdf document
\usepackage{tocloft}						%
\usepackage[noabbrev]{cleveref}				% clever refences (ranges,...)

% Color settings for the pdf file
    \hypersetup{
        hidelinks,
    %   backref=true,                       % bibliography links to text
    % 	bookmarks=true,
     	pdftitle={PDF title},
    	pdfauthor={Author},          %<<----TODO: author
    	colorlinks=false,                   % we use text colors
    %	linkcolor=hiddenLink,               % color for links...
    %	citecolor=hiddenLink,
    %	filecolor=link,
    %	menucolor=hiddenLink,
    %	urlcolor=hiddenLink,
    	bookmarksnumbered=true,             %
    	linkbordercolor=1. 1. 1.,           % frame around links->white
    	urlbordercolor = 1. 1. 1.           % frame around urls->white 
}

\usepackage[style=numeric,natbib=true,backend=bibtex8]{biblatex}
\addbibresource{./backmatter/licbib170505.bib}%{./backmatter/ref170411.bib}
%------------------------
%\usepackage[backend=biber,                  % configures bibliography, use biber
%            style=alphabetic,
%            maxcitenames=3,
%            maxbibnames=5,
%            firstinits=true,
%            url=false,
%            isbn=false,
%            autolang=other
%            ]{biblatex}
%%-------------------------
%%    \appto{\bibsetup}{\raggedright}        
%    \DeclareSourcemap{% used for mapping language to hyphnation in bibtex 
%        \maps{
%            \map{
%                \step[fieldsource=language, fieldset=langid, origfieldval, final]
%                \step[fieldset=language, null]
%            }
%        }
%    }
%}            


%\bibliography{./backmatter/reflib170328} 

\usepackage[nomain,%
            nopostdot,%
            acronym,%
            toc,%
            nonumberlist,
            translate=babel]{glossaries}		% glossary package for acronyms
\makeglossaries


%%% TEST/DRAFT packages, disable in final! ===================

%\usepackage[l2tabu, orthodox]{nag}	% added 21/01/14 by DK
\usepackage[colorinlistoftodos]{todonotes}
\usepackage{blindtext}
%\usepackage{layouts}
%%% Other Useful Packages (Added 02/05/12 by WW)

%\usepackage{lineno} %add line numbers to draft copies
%\linenumbers
%\setpagewiselinenumbers

%%% END TEST/DRAFT packages ==================================

%\usepackage{mdwlist} % remove white space between lists


% -- Layout ------------------------------------------------
% .. page size .............................................
\setstocksize{240mm}{169mm} % this is the real size
\settrimmedsize{\stockheight}{\stockwidth}{*}

% trimmarks (use for measuring when printing A4 paper)
%\setstocksize{297mm}{210mm}
%\trimFrame
%\settrimmedsize{240mm}{170mm}{*}
%\settrims{2cm}{2cm}

\setbinding{5mm}
\settypeblocksize{*}{115mm}{1.6} % davkra 2015-02-24
% spine, edge, ratio
\setlrmargins{20mm}{*}{*}
% upper lower, ratio
\setulmargins{*}{*}{1.2}


% .. page layout ...........................................
% .. headers and footers
\makepagestyle{miunlic}
\makeevenhead{miunlic}{\small\leftmark}{}{}
\makeoddhead{miunlic}{}{}{\small\rightmark}
\makeevenfoot{miunlic}{\small page | \thepage}{}{}
\makeoddfoot{miunlic}{}{}{\small page | \thepage}

\makeatletter % because of \@chapapp
\makepsmarks {miunlic}{%
  \nouppercaseheads
  \createmark{chapter}{both}{shownumber}{\@chapapp\ }{. \ }
  \createmark{section}{right}{shownumber}{}{. \ }
  \createmark{subsection}{right}{shownumber}{}{. \ }
   %TODO: decide subsub or subsection
  \createplainmark {toc}{both} {\contentsname}
  \createplainmark {lof}{both} {\listfigurename}
  \createplainmark {lot}{both} {\listtablename}
  \createplainmark {bib}{both} {\bibname}
  \createplainmark {index}{both} {\indexname}
  %\createplainmark {glossary}{both} {\glossaryname}
}
\makeatother

% .. adjust plain style too
\makeevenfoot{plain}{\small page | \thepage}{}{}
\makeoddfoot {plain}{}{}{\small page | \thepage}

% .. table of contents .....................................
\renewcommand{\cftchapterfont}{\scshape}
\renewcommand{\cftchapterpagefont}{\scshape}
%\renewcommand{\cftsectiondotsep}{\cftnodots}

% .. numbering depth
\setsecnumdepth{subsection}
\maxsecnumdepth{subsection}

% .. numbering depth
\maxtocdepth{section}
\settocdepth{section}
% .. text separators........................................
% .. parts
%\renewcommand{\partnamefont}{\LARGE}
%\renewcommand{\partnumfont}{\scshape\LARGE}
%\renewcommand{\parttitlefont}{\color{miunBlue}\scshape\huge}
%\renewcommand{\cftpartfont}{\color{miunBlue}\scshape}
%\renewcommand{\parttitlefont}{\scshape\huge} % minimize color usage 21/01/14 by DK
%\renewcommand{\cftpartfont}{\scshape}
%
%\renewcommand{\cftpartpagefont}{\scshape}

% .. chapter style
\headstyles{bringhurst}
%\chapterstyle{pedersen}
%\chapterstyle{chappell}


% these commands go into chapter titles
\def\Vhrulefill{\leavevmode\leaders\hrule height 0.7ex depth \dimexpr0.4pt-0.7ex\hfill\kern0pt}

\def\nbrcircles{100}%100! 247 377
\def\outerradius{10mm} 
\def\deviation {.9}
%\def\fudge {.62}
\def\fudge {.27}

% the chapter style
\makeatletter 
\makechapterstyle{dots}{%
    \chapterstyle{default}
    \renewcommand*{\chapterheadstart}{}
    \renewcommand*{\printchaptername}{%
        \centerline{\parbox{0.5cm}{\Vhrulefill} \chapnumfont{\@chapapp\ \thechapter} \parbox{0.5cm}{\Vhrulefill}}}
    \renewcommand*{\chapternamenum}{}
    % \renewcommand*{\chapnumfont}{\normalfont\scshape\MakeTextLowercase}
    \renewcommand*{\chapnumfont}{\normalfont\sffamily}
    \renewcommand*{\printchapternum}{}
    \renewcommand*{\afterchapternum}{\vspace{2mm}
        %
        %\par\centerline{\parbox{0.5in}{\hrulefill}}\par
    }
    \renewcommand*{\afterchaptertitle}{\par\nobreak\vskip0.5\midchapskip}
    \renewcommand*{\printchapternonum}{%
        \vphantom{\chapnumfont \@chapapp 1}\par
        \parbox{0.5in}{}\par}
    \renewcommand*{\chaptitlefont}{\normalfont\sffamily\Huge}
    \renewcommand*{\printchaptertitle}[1]{%
        \centering \chaptitlefont\MakeTextUppercase{##1}\\
    }}
\makeatother
\chapterstyle{dots}

% .. sections
%\setsecheadstyle{\scshape\large}
\setsecheadstyle{\large\sffamily}
\setsubsecheadstyle{\itshape\sffamily}
\setsubsubsecheadstyle{\itshape\sffamily\small}
% .. captions
\captionnamefont{\sffamily\footnotesize}
\captiontitlefont{\sffamily\footnotesize}

%\subcaptionlabelfont{\sffamily\footnotesizefd}
%\subcaptionfont{\sffamily\footnotesize}
%\changecaptionwidth
%\captionwidth{0.75\textwidth}

\captionsetup[subfigure]{font={scriptsize,sf}}
\captionsetup[figure]{width=\textwidth}

\setfloatadjustment{figure}{\centering}
% finalize it!
\checkandfixthelayout

% !TeX root = main.tex
% !TeX spellcheck = en_GB


\hyphenation{Me-trop-olis}


%\newcommand{\galpha}{\ibygr{a}}
%\newcommand{\gbeta}{\ibygr{b}}
%\newcommand{\ggamma}{\ibygr{g}}
%\newcommand{\gdelta}{\ibygr{d}}
%\newcommand{\gDelta}{\ibygr{D}}
%\newcommand{\gepsilon}{\ibygr{e}}

\newcommand{\galpha}{\textalpha}
\newcommand{\gbeta}{\textbeta}
\newcommand{\ggamma}{\textgamma}
\newcommand{\gdelta}{\textdelta}
\newcommand{\gDelta}{\textDelta}
\newcommand{\gepsilon}{\textepsilon}

\newcommand{\mff}{\SI{55}{\micro\meter}}
\newcommand{\mht}{\SI{110}{\micro\meter}}
\newcommand{\threeh}{\SI{300}{\micro\meter}}
%\newcommand{\FIXME}{\noindent\hrulefill FIXME \hrulefill}
\newcommand*\FIXMEl{\color{red}\par\noindent\raisebox{.8ex}{\makebox[\linewidth]{\hrulefill\hspace{1ex}\raisebox{-.8ex}{FIXME}\hspace{1ex}\hrulefill}}\color{black}}
\newcommand{\FIXME}{\color{red}FIXME\ \color{black}}
% faster medipix commands
\newcommand{\mpx}[1][]{\textsc{Medipix#1}}
\newcommand{\tpx}[1][]{\textsc{Timepix#1}}
% neutron detector
%\newcommand{\tib}{\ce{TiB2}}
\newcommand{\tib}{\ce{TiB2}}
% Define the command. Note that the input and output folders are static!
%\newcommand{\includetikz}[1]{%
%	\input{figures/#1.pgf}%
%}

% inputtikz command for figures
\newcommand*{\inputtikz}[2]{%
\resizebox*{#1\textwidth}{!}{%
\input{figures/#2}}}

% we need this for correct label references to papers
\newcounter{dummy}

% -- empty page commands --
%\newcommand\emptypage{\newpage\null\thispagestyle{empty}\newpage}   % inserts empty page (empty)
%\newcommand\emptypagenum{\newpage\null\thispagestyle{plain}\newpage}% inserts empty page (plain)

% commands only for editing:
% do not externalize the \todo command
\makeatletter
\renewcommand{\todo}[2][]{\tikzexternaldisable\@todo[#1]{#2}\tikzexternalenable}
\makeatother



\newcommand{\todocite}[1]{\todo[color=yellow, author=cite]{#1}}


%\glsdefmain{main}
%\makeglossaries
%\input{./backmatter/glossary} %folder added by WW 02/05/12
%% !TeX root = ../main.tex
\newacronym{lwc}{LWC}{Liquid Water Content}
\newacronym{mvd}{MVD}{Median Volume Diameter}
\newacronym{led}{LED}{Light Emitting Diode}
\newacronym{cmos}{CMOS}{Complementary Metal Oxide Semiconductor}
\newacronym{log}{LoG}{Laplacian of Gaussian}
\newacronym{dii}{DII}{Droplet Imaging Instrument}
\newacronym{cdp}{CDP}{Cloud Droplet Probe}
\newacronym{smhi}{SMHI}{Sveriges Meteorologiska och Hydrologiska Institut, (Sweden's Meteorological and Hydrological Institute)}
\newacronym{arome}{AROME}{Application de la Recherche à l'Opérationnel à Méso Echelle, (Application of Research to Operations at Mesoscale)}
\newacronym{nwp}{NWP}{Numerical Weather Prediction}
\newacronym{lam}{LAM}{Limited Area Models} %folder added by WW 02/05/12


%
% \\\\\\\\\\\\\\\\\\\\\\ THE THESIS ////////////////////////
%

% !TeX root = main.tex
% !TeX spellcheck = en_GB
%: title
%TODO:
\title{Measuring Water Droplets to Detect Atmospheric Icing}
\author{Staffan Rydblom}
\date{\today}

\newcommand{\theSupervisor}{Associate Professor Benny Thörnberg,\\ Professor Mattias O'Nils}
\newcommand{\theThesisNumber}{XX}
\newcommand{\theISSN}{XXXX-XXXX}
\newcommand{\theISBN}{XXXX-XXXX} %added by WW 02/05/12 note: you can request these numbers from the library at http://www.bib.miun.se/eng/researchers/publishing/theses/preparations
%\newcommand{\subtitle}{Monte Carlo and Charge transport simulation in Medipix detector systems}


%-----------------------------------------------------------
% DON’T TOUCH THIS!
%\newlength{\drop}
\newcommand*{\titleCC}{\begingroup
%\drop=0.0\textheight
%\vspace*{\drop}
\centering 
{Thesis for the degree of Licentiate of Technology\\
Sundsvall 2017}\\[1.5cm]
\par
{\huge \thetitle}\\[0.25cm]
%{\Large\subtitle}\\[1.25cm]

\par
{\Large \theauthor}\\[.75cm]
\par
{Supervisor: \theSupervisor}\\[.75cm]
\par
{Department of Electronics Design, in the\\
Faculty of Science, Technology and Media\\
Mid Sweden University, SE-851 70 Sundsvall, Sweden}\\[0.75cm]
\par
{ISSN \theISSN}\\[0.75cm]
{ISBN \theISBN}\\[0.75cm] % added by WW 02/05/12
\par
{Mid Sweden University Licentiate Thesis \theThesisNumber}\\[0.75cm]
\vfill
%{\includegraphics[width=0.4\textwidth]{./figures/ElectronicsDesign_logo}}\\[0.2cm] % added by WW 02/05/12 note: if you use this logo, you might want to reduce the MIUN logo to width=0.4\textwidth
{\includegraphics[width=0.5\textwidth]{./figures/MIUN_logotyp}}\\[0.5\baselineskip] %folder added by WW 02/05/12
%\vfill
\clearpage
\endgroup}


\pagestyle{miunlic}

\begin{document}
% -- frontmatter -------------------------------------------

\frontmatter
% .. titles ................................................
%todo remov in final
%\listoftodos
\newpage
\thispagestyle{empty}
\titleCC
% !Tex root = ../main.tex
\thispagestyle{empty}

{\raggedright
Akademisk avhandling som med tillstånd av Mittuniversitetet i Sundsvall framläggs till offentlig för avläggande av licentiatexamen i elektronik \textbf{onsdagen den 1 november 2017}, klockan \textbf{??.00} i sal \textbf{??}, Mittuniversitetet Sundsvall. Seminariet kommer att hållas på engelska.

\vfill

{\large\thetitle}

\vspace{1cm}

\theauthor

\vspace{1cm}

\textcopyright \theauthor, 2017
\vspace{1cm}

Electronics Design Division, in the
Faculty of Science, Technology and Media
Mid Sweden University, SE-851 70 Sundsvall
Sweden

\vspace{1cm}

Telephone: +46 (0)60 148422

\vspace{1cm}

Printed by Kopieringen Mittuniversitetet, Sundsvall, Sweden, 2017
} %folder added by WW 02/05/12
% .. dedication ............................................
\cleardoublepage
\null\vspace{\stretch{2}}
\begin{flushright}
	% !TeX root = ../main.tex
% !TeX spellcheck = en_GB
%: dedication

\vspace{4cm}
For Sara, Alva and Ludvig.
%x=16*sin^3(t)
%y=13*cos(t)-5*cos(2*t)-2*cos(3*t)-cos(4*t)

\begin{figure}[h]
\begin{flushright}
  \includegraphics[width=0.20\linewidth]{figures/heart}
 \end{flushright}
\end{figure}

 %folder added by WW 02/05/12
\end{flushright}
\vspace{\stretch{1}}\null

% .. start english abstract ................................
\selectlanguage{english}
\abstractintoc
\abstractnum

% !TeX root = ../main.tex
% !TeX spellcheck = en_GB
%: Abstract english
\begin{abstract}
This thesis is about...
\end{abstract} %folder added by WW 02/05/12
\clearpage

% .. start swedish abstract ................................
\selectlanguage{swedish}
\abstractintoc

% !TeX root = ../main.tex
% !TeX spellcheck = en_SV
%: Abstract swedish

\begin{abstract}

Den här avhandlingen beskriver hur en metod för att mäta storlek och koncentration av vattendroppar i atmosfären. Målet är att hitta en kostnadseffektiv teknik som kan förutsäga isbildning.

Isbildning som orsakas av atmosfäriskt vatten kan vara ett allvarligt problem för infrastruktur som t.ex. kraftledningar, vägar och flygtrafik. Omkring en tredjedel av världens vindkraft finns i klimat som definieras som kalla, där nedisning av rotorbladen är en av de största utmaningarna.

Nedisningsprocessen är komplex och resultatet beror på en kombination av strukturen eller vingens aerodynamiska form, den förbipasserande luftens hastighet, luftens och ytans temperatur, eventuell blandning av snö och vatten, koncentrationen av flytande vatten och vattendropparnas storleksdistribution.

Den valda mätmetoden baseras på skuggfotografering med en \gls{led} som bakgrundsbelysning samt digital bildbehandling. Ett prototypinstrument har konstruerats med hjälp av kommersiellt tillgängliga komponenter. Valet av komponenter möjliggör låg produktionskostnad av en serieproducerad variant av detta instrument. Exempel på applikationer för ett kommersiellt instrument baserat på denna teknik är villkorsmätning för nedisning i realtid samt assimilation och verifikation av data i numeriska vädermodeller.

Arbetet som presenteras visar att mätning av storlek och koncentration av vattendroppar med hjälp av skuggfotografering kan användas för validering av numeriska vädermodeller (\gls{nwp}) och andra instrument. Noggrannheten hos storleksmätningen är hög. Noggrannheten av koncentrationsmätningen har potential att bli hög eftersom varje droppe får en individuellt kalibrerad mätvolym. Precisionen hos mätinstrumentet beror till största delen på hur många bilder och vattendroppar som används för att skapa varje mätvärde. Instrumentets prestanda för att mäta i realtid begränsas av kamerans bildhastighet och bildbehandlingstiden samt vilken mätvärdesprecision som önskas.

\end{abstract} %folder added by WW 02/05/12
\clearpage

% .. start toc,lof,lot .....................................
\selectlanguage{english}
\tableofcontents
\clearpage
\listoffigures
\clearpage
\listoftables
\clearpage %line added by WW 02/05/12
% !TeX root = ../main.tex

%\chapter{List of Papers}

\thispagestyle{plain}

\chapter*{List of Papers}
%\noindent {\Huge\bfseries\sffamily List of Papers}\\
\refstepcounter{dummy}
\addcontentsline{toc}{chapter}{List of Papers}
\vspace{20pt}

\noindent This thesis is based on the following papers, herein referred to by their Roman numerals:  

% paper name in a short command
\newcommand{\paperone}{This is the title of my first paper}


% authors in a short command
\newcommand{\authorone}{D.Krapohl, H.-E. Nilsson, S.~Petersson, S.~Pospisil, T-~Slavicek and G.~Thungström}


%some journals in short
\newcommand{\jinst}{Journal of Instrumentation}
\newcommand{\nss}{Nuclear Science Symposium Proceedings}
\newcommand{\spie}{Proceedings of SPIE}

\begin{description}[style=nextline]
    \item[Paper I]
    \paperone \\ 
    \authorone, \jinst, 2011\dotfill \pageref{pap:paper1}
    
    \item[Paper II]
 
    \item[Paper III]
    
    \item[Paper IV]

    \item[Paper V]
    
    \item[Paper VI]
   
    \item[Paper VII]

    \item[Paper VIII]



\end{description}

 %line added by WW 02/05/12
\microtypesetup{protrusion=true}
%\printglossary[style=long4colheader, title=Glossary]


% .. start acknowledgment ..................................
% !TeX root = ../main.tex
% !TeX spellcheck = en_Gb

\chapter{Acknowledgements}

I would like to express the greatest gratitude to my supervisor, Benny Thörnberg, for his trust and encouragement during hard times when things did not turn out the way they were expected.
I would also like to thank Mattias O'Nils and Claes Mattsson at the Electronics Department for making the project possible and giving me the possibility to finish the thesis. A big thank is also given to the Swedish Energy Agency for funding most of the work, Patrik Jonsson, Björn Ollars and Olof Carlsson at Combitech for their invaluable work with integrating the CDP and Esbjörn Olsson at SMHI for sharing his knowledge about the weather physics and his work with the NWP model data.
The greatest love and gratitute towards my wonderful family that have sacrificed and supported me in beginning, fulfilling and finishing this work. Without you this would not have been possible.
Special thanks to Lotta Frisk for chasing my signatures, Fanny Burman for the dances, Carolina Blomberg for the skis, Christine Grafström for the run and Maria Rumm for SMART inspiration.
I would also express my gratitude to my mentor Magnus Göran Tungström, David Kraphol, Kent Bertilsson, Henrik Andersson, Javier Brugés, Igor Fedorov, Jan Thim, Enkeleda Balliu, Muhammad Abu Bakar, Amir Yousuf and Johan Sidén for support and fruitful discussions.
Lisa Velander for language correction.
All the people at Mid Sweden University. %folder added by WW 02/05/12

% -- start main content ------------------------------------
\mainmatter
% !TeX root = ../main.tex
% !TeX spelling = EN_gb


\chapter{Introduction}
Measuring and controlling the properties of a fog of water droplets has been the interest and focus of many studies during at least half a century. Applications range from atmospheric studies, aircraft safety to military and commercial applications. 

Icing caused by freezing atmospheric water can be a significant problem in cold climates, affecting infrastructure such as wind turbines, power lines, and road and air traffic. With the increasing importance of electric power generated by wind, there is a renewed demand predictions of icing on wind turbines. Icing on the blades of a turbine lowers the efficiency, increases noise and may force the turbine to a complete stop [1-4]. Aircraft, power lines or any other weather exposed structure share this problem. Therefore big efforts have been made to create models for how the ice is formed [5, 6] and how it can be included in weather prediction models [7, 8].

This thesis is an exploration of a cost effective and robust method to measure atmospheric liquid water in order to predict icing on ground based structures.

The measurement method is based on digital image processing in a shadowgraph system using LED light as background illumination. A prototype instrument is built and initially tested using a fog chamber. It is calibrated using a micrometer dot scale with circular discs printed on a silicon glass. Polymer microspheres are also used for size measurement verification.

The instrument is constructed using standard components with the intent of viewing possibilities of low cost volume production. It may e.g. be used for real-time icing condition measurement, or remote meteorological data collection. 

Following is a brief description of related work and an introduction to the attached papers.

\section{Related Work}

In 1970, Knollenberg [23] described an electro-optical technique to measure cloud and precipitation particles using a laser illuminated linear array of photo detectors. The photo detectors are used to make a two-dimensional image of the particles’ shadows as they pass the light beam. Systems based on this technique are called Optical Array Probes (OAP) or two-dimensional imaging probes. Later development of this technique includes using image sensors to save gray scale images of the detected particles [17, 22].

Light scattered from a focused laser  [15, 24] is one common technique to measure single particles. A laser beam is used to illuminate passing particles. When a particle is detected, its size is determined by comparing the variations in light with a pattern derived from the Mie scattering solutions [25].

The OAPs and the light scattering spectrometers each have some advantages over the other technique depending on the nature of the aerosol. Instruments for airborne use have been developed that combine several techniques into one single probe [26] for accurate measuring of LWC and MVD [16].

A similar but different optical technique for measuring water droplets is based on in line holography [27]. In principle this is a two-dimensional shadowgraph imaging system that use laser background illumination to create images of the diffraction patterns created by the passing particles. These patterns are measured to reconstruct images of the particles. This is a fairly calculation intensive process, but which may be one of the reasons why instruments based on this technique are not so common [28].

Sizing of the droplets using Mie spectrometry is a complex operation even for coherent light [29], and although it is possible to study Mie scattering from white light [30], the complexity, small droplet size range and sample volume makes it less attractive in this application. The optical resolution is usually too low. Therefore it is also difficult to determine the exact particle size and the usable depth of field for incoherent shadowgraph systems. The shadow from a particle can e.g. appear smaller or larger when out of focus.

Shadowgraph imaging of particles using incoherent illumination instead of laser has been tried e.g. in particle shadow velocimetry (PSV) [31, 32], or spray characterization [33]. Quantitative and comprehensive studies of other droplet measurement techniques exist [24, 28, 34].

Kuhn et al (2012) described a method to characterize ice particles by Fourier shape descriptors (Granlund 1972). The system uses a microscope-like technique to achieve a high resolution level, in the order of one micrometer. Perhaps this method can be used also for larger scale objects like snowflakes. The system described by Kuhn et al has a field of view of 200x150um and a depth of field of approximately ten micrometers.



 %folder added by DK 02/09/14
%
% !TeX encoding = UTF-8
% !TeX root = ../main.tex
% !TeX spelling = en_GB
% !dsfaTeX program = latexmk

\chapter{Theory}
\label{chap:theory}

This chapter begins with an introduction to the physics behind the forming of ice on structures. The next two sections briefly describe the measurement parameters and some practical problems of measuring icing. The last section is about scattering of light by small particles.

\section{Forming of Ice}

Cold climate areas, according to the definition by IEA \cite{iea2017} are regions that experience frequent atmospheric icing or periods with temperatures below the operational limits of standard IEC 61400-1 ed3
wind turbines. Atmospheric icing is the period of time where atmospheric conditions are present for the accretion of ice or snow on structures \cite{iea2017}.

An icing event can be divided into three phases: the incubation, the accretion and the persistance/ablation. Meteorological icing, which is the main interest of this thesis, is the period during which the meteorological conditions (temperature, wind speed, liquid water content, \gls{dsd}) allow ice accretion. The incubation is the time in meteorological icing before the accretion starts. The length of each of these phases, and the severeness of the icing depends on a combination of the aerodynamic shape and temperature of the structure or airfoil, the velocity of the air and its contained water, the air temperature, the mixing of snow and water, the concentration of liquid water and the \gls{dsd}.

Whether a particle is likely to follow the flow or collide depends on the flow velocity, the size and shape of the obstacle and the density and drag coefficient of the particle. This relationship is known in fluid mechanics as the Stokes number ($Stk$). Small droplets or particles with $Stk \ll1$ may continue with the airflow around the profile, while large droplets or particles with $Stk \gg 1$, due to their inertia, collide with the structure. A supercooled droplet colliding with a structure is likely to freeze upon impact. Figure \ref{fig:freezedrops} illustrates the difference between large and small supercooled droplets passing an aerodynamic profile.

\begin{figure}%[h]
\centering\includegraphics[width=0.8\linewidth]{./figures/freezing_droplets.jpg}
\caption{Supercooled water droplets on collision course with an aerodynamic profile.}
\label{fig:freezedrops}
\end{figure}

Icing is a slightly different problem for wind turbines than for aircraft. Unlike aircrafts, the turbine is stationary and cannot stop the icing, or de-ice by moving to a different position. The same conditions may persist for days or weeks. The turbine blade moves concentrically making the tip of the blade the fastest, and highest located moving part. The remote and exposed position makes it difficult to detect ice directly when it is initially formed \cite{homo2006}.

\section{Liquid Water Content and Median Volume Diameter}

The liquid water content (LWC) and the water droplets median volume diameter (MVD) are parameters that can be used to predict or model icing. The  MVD is given at the point where half of the total volume of liquid content in a fixed air volume consists of droplets with larger diameters, and half with smaller diameters. The MVD has been assumed to give the best approximation to the spectrum of diameters in a \gls{dsd}, when considering the collision efficency \cite{fins1988}. To estimatie the amount of icing created by supercooled water droplets, the MVD has been shown to be a good indicator in most cases \cite{makk2000}. The MVD as approximation to the \gls{dsd} can be used to simulate ice accretion on wind turbines \cite{dier2011}.

In practice, the LWC and MVD are rarely measured at a planned or existing wind turbine \cite{parent2011, makk1992}. Measuring these properties accurately and frequently would be an advantage for the planning of new wind mill farms or for the application of anti-icing arrangements on existing power stations. It may be of particular interest as input to weather prediction models, by which both LWC and MVD can be computed \cite{thomp2009, nyga2011}. In combination with information about the aerodynamic properties of the wind turbine, it can give more accurate predictions of icing or even result in better design of wind turbines and anti-icing methods.

While icing caused by large supercooled droplets, with diameters from approximately 50 μm to more than 1000 μm, is often considered severe due to its shape and quick build-up, icing may occur even with droplets as small as 5 μm \cite{sand1984, cob2001, homo2010}. In most cases though, icing is caused by cloud droplets measuring between 10 μm and 30 μm in diameter \cite{makk1992, cob2001}.

Although optical imaging and other techniques for measuring aerosol properties are continuously improving, the choice of instrument is still very much dependent on the application’s requirements \cite{ide1999, baum1983, baum2011, kulk2011}. An instrument for measuring icing parameters for wind turbines should be able to detect supercooled cloud droplets as small as five micrometer and determine an accurate measure of LWC. Since measurements are needed in multiple remote locations, it should also be affordable, reliable, and, ideally it should be possible to place it near the highest point of the turbine \cite{homo2006}.

\section{Practical Problems with Measuring Atmospheric Water}

The varying nature of atmospheric water particles mentioned in the previous chapter makes it very difficult to measure all kinds and sizes of particles using a single instrument. Therefore, atmospheric aerosol studies are often done with one or several instruments combining different techniques. Each technique with its own limitations and problems. It is also difficult to find a reference sample with the same but known properties as the water. Free floating water droplets are affected by gravity, they eventually collide and coalesce, evaporate or stick to adjoining  surfaces. This makes it difficult to measure and find out the physical properties without affecting the sample.

The size of water droplets range from a few micrometers to several mm in diameter. An imaging instrument is limited by the optical system's resolution, its field of view and the usable depth of field or the measuring range.

Many existing instruments suffer from errors caused by the instrument itself during sampling, e.g. when droplets get stuck on the inlet \cite{spie2012}. At higher wind speeds particles shatter into smaller droplets or bounce on the supporting structure making up the instrument \cite{cohen1991,field2006}. All droplets or particles approaching an obstacle are affected by the change in pressure and wind direction surrounding it, a fact which complicates measuring the concentration of particles in unaffected air. Measurement probes working by extraction of air using a mechanical air pump would expect a loss of particles with large Stokes numbers. Ideally, the measuring device should be designed to have as little effect on the free flow of particles as possible \cite{baum2011}.

An example of an instrument that makes use of the fact that supercooled water droplets will freeze upon impact, is the rotating cylinders used by Makkonen \cite{makk1992}. By exposing cylinders of different diameters, depending on the amount of ice accumulated on each cylinder, the MVD and LWC can be calculated using a theoretical model. While this technique provides an alternative to the single particle measurements its drawback is that it requires a certain extent of manual operation, in addition to the fact that only freezing water can be measured.

Measuring particles via aircraft is complicated by the high air speed. The sample is affected by the change in pressure surrounding the aircraft and by particles hitting parts of the probe, splintering or changing direction, causing anisokinetic sampling \cite{baum2011}. An instrument fixed to the ground on the other hand is affected by the wind speed relative to the ground. This means that it needs to be directed in the direction of the wind. Particles may also enter the measurement zone from different directions depending on their Stokes number, which has been shown to have an effect on the measured liquid water content \cite{henn2013}.

Instruments based on Mie calculations of light scattering sometimes struggle to deal with with a non-linear relation between scattering response and diameter. Aliasing in the sample bin resolution can lead to spikes in the \gls{dsd}. Particularly interesting is the 10 to 15 μm range, where two particles with diameters differing more than one micrometer can have the same scattering intensity response \cite{dye1984,spie2012,bohr2008}.

\section{Light Scattering and Absorption}

In visible light, water is almost transparent. This means that the imaginary part of the refractive index, i.e. the absorption, is very small, while for some wavelengths it increases many times. This fact is used in two-color lidar measurements \cite{west2010}. The real part of the refractive index for water is much more stable; approximately 1.3 in the visible to near infrared range \cite{hale1973, kou1993}. 

For a shadowgraph system, it is possible to assume that the refractive index of air is equal to one. A droplet works as a spherical lens with a very short focal length. Exposed to a background illumination it will scatter almost all of the light that reaches the droplet, causing a shadow that appears as a black disc except for the light passing straight through the center. Some of the light will also be absorbed, but the absorption of light by a single water droplet is negligible due to its small volume.

When the light source is large compared to the size of the droplets, as in the case of using a collimated LED, the intensity of the center Arago spot caused by Fresnel diffraction is small \cite{reis2017}.

The combined effect of scattering and absorption is the extinction \cite{bohr2008}. Due to this combined effect, clouds look nontransparent from a distance. Measuring the light's extinction is possible, e.g. by using a Raman LIDAR \cite{ans1990}.

 %folder added by WW 02/05/12
%
% !TeX root = ../main.tex
% !TeX spelling = en_GB
% !TeX program = pdflatex

\chapter{Materials, Methods and Unpublished Results}
\label{chap:methods}

The work begun with some initial tests, in which different light sources and illumination angles were investigated. It was decided to use a shadowgraph system with background illumination. The experimental setup was tested using different water droplet generators and a test target consisting of micrometer sized lines and dots. Analyzing the optical system and testing different segmentation algorithms we found a simple way to define the sample volume from a single image. Using the Laplacian of Gaussian edge detection, which principally is a second derivation of the image intensity gradient, and a suitable threshold, we can create closed curves around objects where the edge is in or near focus. To test the ability of measuring concentration we needed to build a weather protected prototype that could be used in parallell with a second instrument. The prototype needed to be fully automatic, able to analyze images in real time 24 hours a day during several months and store the results in a compressed format. To calibrate the size measurement and the measurement range, the previously mentioned dots of different sizes were used. The size measurement was also verified using distributions polymer microspheres, applied by blowing compressed air through a glass dispenser.

\section{Image Segmentation}

Detection can be done in several ways. The simplest method is to use a threshold at a fixed level and define the edge as the transition between above and below that threshold \cite{gonz2002}. This technique is Other techniques use different methods to measure the gradient of the change in intensity \cite{canny1986,marr1980}.

\section{Optical Characteristica}

As simplificaiton we describe the measured droplets as spherical lenses made of pure water at a constant temperature, surrounded by air. 

\subsection{Light Spectrum}

Although water is a good absorber of electromagnetic ratiation in most spectral wavelenghts except for the visible, the volume of water droplets is too small for the absorbtion to be measurable. In visible light, the water droplet can be regarded as a spherical lens with a very short focal length, thus spreading most of the light in diverging directions. Since the used camera is specifically designed for visible and near infrared, we tested and compared two different wavelenghts, 455 nm and 850 nm using the same optical setup. The shorter wavelength gave sharper images.

\begin{figure}
\centering
\begin{minipage}{.5\textwidth}
  \centering
  \includegraphics[width=.6\linewidth]{figures/compare455nmdot}
  %\caption{A subfigure}
  %\label{fig:compare455nmdot}
\end{minipage}%
\begin{minipage}{.5\textwidth}
  \centering
  \includegraphics[width=.6\linewidth]{figures/compare850nmdot}
  %\caption{A subfigure}
  %\label{fig:compare850nmdot}
\end{minipage}
\caption{Same dot viewed in different background light. 455 nm (left) and 850 nm (right).}
\label{fig:comparedots}
\end{figure}

One would expect diffraction patterns depending on the spectral bandwith of the light source. The narrower band width the stronger the diffraction would be. A spectrum analyze of the LED showed that the coherence length of the blue LED is about 6.8 μm. Therefore the spectrum was measured using a spectrum analyzer. The result can be seen in Figure \ref{fig:ledspectrum}

\begin{figure}%[ht]
\centering\includegraphics[width=0.6\linewidth]{figures/spektralanalys_mightex455nm}
\caption{Spectrum of the Mightex 455nm LED using two different driving currents. The band width is slightly wider for the higher current.}
\label{fig:ledspectrum}
\end{figure}

\subsection{Laser Light}

Using coherent light for imaging means that the effects of interference patterns will dominate the image of small particles. This is e.g. used when reconstructing images in holographic imaging instruments mentioned in section \ref{sec:relwork}. Laser illumination was tried used using two wavelengths, 450 and 850 nm. The images using laser illuminating the sample from an angle 

\subsection{Image Noise}

The total image noise was first measured using the whole image using the 455nm LED light, resulting in a variation coefficient of about nine percent.

The idea came that we should be able to measure the signal to noise relation, if the shadow image of the droplet is represents the signal. This signal to noise ratio (SNR) could possibly be used to increase the accuracy of the measurement. A function was created that calculates the noise level locally around each analyzed droplet image. A correlation could be seen between the SNR and the corresponding measuring range, resulting in an increased sampling volume for higher SNR. This relation has not been implemented in the LWC calculation for the prototype instrument yet.

The SNR for a ten micrometer dot image was compared for the different available gain level settings in the image sensor. This measurement was done using both the 455 and the 850 nm illumination. The result can be seen in Figure \ref{fig:noisegain}.

\begin{figure}%[ht]
\centering\includegraphics[width=0.6\linewidth]{figures/NoiseGain}
\caption{SNR for a ten micrometer dot image using two wavelengths and seven gain level settings. Shorter wavelength and lower gain gives less noise.}
\label{fig:noisegain}
\end{figure}

\subsection{Ambient Light}

In daylight, there is always some ambient light. Although the camera is never aimed direclty at the sun and the system make use of a telecentric lens we wanted to be sure that this light is not enough to affect the measurement. To get an idéa about the amount of ambient light the camera and lens was tested alone in daylight, using a fog generator in front of the lens to reflect light into the lens. 

The shortest possible exposure time according to the camera specification is 0.038 ms, slightly depending on other camera settings. For each measurement 152 images were captured, with increasing exposure time from 0.040635 to 1.988535 ms. A delay of 1 second was set between each image. Figure \ref{fig:ambientlight} shows the mean pixel value and the standard deviation of the value for one of the measurements at 22000 lux ambient light.

\begin{figure}%[ht]
\centering\includegraphics[width=0.6\linewidth]{figures/Amblight22000lux}
\caption{Mean pixel value and standard deviation of ambient for a light measurement at 22000 lux.}
\label{fig:ambientlight}
\end{figure}

Using this setup it was found that an exposure time of at least one second is needed to give a signal that is comparable to the noise level in a normal exposure. In order to image small droplets, the exposure time needs to be less than one μs, preferrably less. This difference is greater than the 8-bit pixel resolution used ($2^8=256$).

\subsection{Light Sensitivity Measurement}

The illumination energy required to get a good exposure for each of the eight gain setting levels is shown in Fig. 9. 34 nJ is required for the area in view (7.8 $mm^2$) for the lowest gain setting, which results in the highest SNR. The area in view is about 8 times smaller than the illuminated area of the collimated LED, which was measured to about 8x8 mm. 

\begin{figure}[ht]
\centering\includegraphics[width=0.75\linewidth]{figures/Energy_per_gain_level2}
\caption{Light energy ($nJ$) for exposure in each of the eight gain levels used. The energy was tuned manually up to a level of high exposure, but not saturating any point in the image.}
\end{figure}


\section{Design Considerations}

\subsection{Speed of Light Flash}

A value of both MVD and LWC can be derived from a series of images and since the number of measured droplets will depend on the concentration, the accuracy and precision will depend on the number of samples from the total population of droplets. 

The illuminative power required to get a good exposure is tested.

\section{Calibration and Validation of the Calibration}

The system is calibrated using a stage micrometer scale with 13 circular dots printed in chrome on a silicon glass. The dots range from 2 to 100 μm in diameter. Each dot is moved linearly in steps of one micrometer in the direction orthogonal to the lenses, thus creating a function where the gradient of the edge depends on the distance from optimum focus. A threshold on the second derivative gradient strength limits the measured particles to be within a specific measuring range. It is important to select the threshold carefully. If the value of the threshold is too low, there will be many false edges in the image. If it is too high, the measuring range will be too small. The difference between two different thresholds, 0.002 and 0.005 is illustrated by Figure \ref{fig:measrangevslogth}.  The measuring range is here defined as the distance in which the edge detection by Laplacian of Gaussian operator and a threshold makes a closed curve in the resulting binary image.

\begin{figure}[ht]
\centering\includegraphics[width=0.75\linewidth]{figures/meas_range_vs_log_th}
\caption{Measuring range (mm) vs. diameter for two different thresholds (0.005 and 0.002).}
\label{fig:measrangevslogth}
\end{figure}

The edge sharpness will affect the position of the edge and the measured shadow slightly. We use the maximum value of the second derivative $P_{i,j}$ for each droplet as a measure of the edge sharpness and include this in the calibration functions, together with the diameter measured from the shadow intensity.

Two second degree approximation surfaces, (\ref{eq:z1}) and (\ref{eq:z2}), are calculated using the “fit” command in Matlab. $z_1$ approximates dot diameters from 2 to 10 micrometers and $z_2$ diameters from 10 to 100 micrometers. x is the maximum second derivative, $d^M$ is the diameter measured from the shadow intensity, $p_{xx}$ and $q_{xx}$ are constants.

\begin{equation} \label{eq:z1}
z_1=p_{00}+p_{10} x+p_{01} d^M+p_{20} x^2+p_{11} xd^M+p_{02} {(d^M)}^2
\end{equation}
\begin{equation} \label{eq:z2}
z_2=q_{00}+q_{10} x+q_{01} d^M+q_{20} x^2+q_{11} xd^M+q_{02} {(d^M)}^2
\end{equation}
By measuring the calibrated microspheres a validation can be made with the expected diameter. 

\subsection{Verification of Dot Size}

The dots on the micrometer scale were measured visually using a Leica microscope connected to a digital camera. This was done to get accurate values of the diameter of the dots for use in the calibration. An example image that was measured can be seen in \cref{fig:50umdot40x}.

\begin{figure}[ht]
\centering\includegraphics[width=0.75\linewidth]{./figures/50umdot40x.jpg}
\caption{The 50 micrometer dot with front illumination imaged using 40x magnifying lens.}
\label{fig:50umdot40x}
\end{figure}

This measurement was done using lenses with two different magnifications: 40x and 100x. All the dots average diameters, except the 100 µm dot were found to be within $\pm$ 0.2 $\mu$ m of their nominal diameter. They are not all perfectly round, but the size accuracy should be good enough to use as calibration reference. The result from this measurement is comprehended in \cref{tab:ref_meas}.

\begin{table}[ht]
\centering
\begin{tabular}{p{0.2\linewidth} p{0.2\linewidth} p{0.2\linewidth} p{0.2\linewidth}}
\hline
\textbf{Nominal Diameter} & \textbf{Excentricity} & \textbf{Diam. 40x} & \textbf{Diam. 100x} \\
\hline
5 & 0.2 & 5 & 4.8 \\
6 & 0.6 & 5.8 & 5.8 \\
7 & 0 & 7 & 6.6 \\
8 & 0 & 8 & 7.9 \\
9 & 0 & 9 & 8.8 \\
10 & 0 & 10.1 & 9.8 \\
25 & 0 & 24.9 & 24.7 \\
50 & 0.4 & 50.1 & 49.8 \\
75 & 0.2 & 75.2 & 74.8 \\
100 & 2.5 & 100 & 98.7 \\
\hline
\end{tabular}
\caption{Micro dot verification measurement. All values are in µm. Eccentricity is here the maximum difference in µm between the smallest and the largest measured diameter.}
\label{tab:ref_meas}
\end{table}


\section{The Shadowgraph System}

The instrument is a shadowgraph system using a monochrome CMOS camera with a 4x magnifying telecentric lens and a LED with a collimating lens illuminating the background. The blue LED is powered by a current driver able to produce short 12 A current pulses. Figure \cref{fig:shadowprinc} shows a sketch of the system. 

The system is mounted in a weather proof shell using two standard camera housings and a separate box for the analyzing computer and power supply. Fig. 2 shows the system mounted in weather proof camera housings. 

\begin{figure}[ht]
\centering\includegraphics[width=0.75\linewidth]{./figures/shadowprinc.jpg}
\caption{Principle of shadowgraphy. 1. Camera. 2. Telecentric lens. 3. Parallell focused light beam. 4. Collimating lens. 5. LED.}
\label{fig:shadowprinc}
\end{figure}

\begin{figure}[ht]
\centering\includegraphics[width=0.75\linewidth]{figures/Foto0169}
\caption{The experimental setup with a dot micrometer scale as test object mounted on a translation stage.}
\end{figure}

\begin{figure}[ht]
\centering\includegraphics[width=0.75\linewidth]{figures/cam_housings}
\caption{The illumination and detector is mounted inside two Bosch camera housings facing each other with heating inside the house and on the front glass.}
\end{figure}

Calibration of true droplet size and measuring range both depends on the measured size of the droplet shadow and the amount of light used for exposure. It is possible to predict both the precision of the measured size and accuracy for measurement of droplet size. 

We used a stage micrometer scale for characterization of the system and simulation of water droplets. This characterization holds true given that the optical silhouette of a droplet is comparable to a dot having equal diameter and being printed on a silicon glass. It is not a new concept and has at least once been proved experimentally, by comparing with beads of glass and water droplets of known sizes \cite{koro1991,koro1998}. The shadow image of water drops of any size will be defined mainly by the diffracted component, as long as the distance between the drop and the lens is much larger than the drop diameter \cite{koro1991,wend2013}. 

The design using a weakly collimated LED that illuminates an area slightly larger than the field of view makes the system quite insensitive to misalignment of the camera and the light source. Temporal or permanent changes in light intensity caused by a minor misalignment is automatically compensated for by continuous measurement of the total exposure level. If the level of exposure is increasing or decreasing, the length of the light pulse is changed correspondingly. The light intensity can also be affected by dirt on the front glass of the housings. 

Since many images will not contain any droplet at all, we can increase the processing speed by sorting out the images that are not containing any interesting information. This is done by constructing an average image from 20 images and use this as a flat-field correction. All new images are compared with this average and if any pixel differs from the average by more than a specific value that is significantly higher than the noise level, the image is analyzed.

Spatial dissimilarities in the light intensity that are not caused by noise are compensated for by calculating the local average intensity of the background around each measured droplet. The size of a droplet is then based on the intensity dip caused by the shadow compared with its local background.

\subsection{Exposure Check and Flash Intensity Adjustment}

Let $I_{i,j}$ depict the two dimensional image captured by the camera. The mean value $\overline{i}$ of all pixels in the image $I_{i,j}$ gives an estimation of the exposure level in the whole image. 

The flash duration is adjusted automatically by the microcontroller at each exposure to keep the exposure level between a low and a high threshold, $th_L$ and $th_H$. The duration is changed in steps of 13 ns corresponding to one clock cycle of the microcontroller. If $\overline{i} > th_L$ and $\overline{i} < th_H$ the image is analyzed. If $\overline{i} \geq th_H$ the flash duration is decreased by steps of 12 ns. If $\overline{i} \leq th_L$ the flash duration is increased. The total flash duration is approximately 250 ns. $th_L$ is here set to 0.7 and $th_H$ is set to 0.8 in a normalized (0,1) dynamic range.

\subsection{Edge Detection and Edge Sharpness}

To detect the intensity changes created by the shadow of a water droplet, an image is processed using the from edge detection theory well known Laplacian of Gaussian (LoG) described by Marr and Hildreth \cite{marr1980}. The method works by looking for zero crossings in the image resulting from calculating $\nabla^2 G\left(x,y\right) * I\left(x,y\right)$. $G\left(x,y\right)$ is a two dimensional Gaussian distribution with standard deviation σ and $\nabla^2$ is the Laplacian operator, defined as the divergence of the gradient in two dimensions.
\begin{equation}
G\left(x,y\right) = \frac{1}{\pi \sigma^2} e^{-\frac{x^2+y^2}{2\sigma^2}}
\end{equation}
\begin{equation}
\nabla^2=\nabla\cdot\nabla=\frac{\delta^2}{\delta x^2} + \frac{\delta^2}{\delta y^2}
\end{equation}
We implement a discrete approximation to $\nabla^2 G\left(x,y\right) \approx \nabla^2 G_{i,j}$ as a 13x13 sized convolution kernel.
\begin{equation}
\nabla^2 G_{i,j} = -\frac{1}{\pi \sigma^4} \left(1 - \frac{i^2+j^2}{2\sigma^2} \right) e^{-\frac{i^2+j^2}{2\sigma^2}}
\end{equation}
By applying the convolution to the normalized image $I_{i,j}$ we get the resulting image $P_{i,j}$.
\begin{equation}
P_{i,j}=\nabla^2 G_{i,j} * I_{i,j}
\end{equation}
$P_{i,j}$ is thus a matrix that contains the second order derivative of the image $I_{i,j}$. 
A spherical object will cause an edge where the intensity change is similar around a spherical object, but dependent on the distance from the optimal focus. Assuming the Gaussian of the image I is twice differentiable at any point $(i,j)$, the maximum (or minimum) of the second derivative includes the amplitude of the first derivative at the point where the second derivative is equal to zero, i.e. where the edge is strongest. This can be intuitively understood when considering that if the edge is sharper, the gradient, or first derivative value is larger, and if the gradient value is larger the rate of change, i.e. the second derivative needs to be larger at each side of the edge. Therefore we store a value of the maximum second derivate, $max(P_{i,j}$, around each analyzed object and use this as a measure of the edge sharpness. This value is then used as input to a calibration function. 
We also construct a new binary image $Q$ in which the pixel value $Q_{i,j}=1$ if any of $|P_{i+1,j}-P_{i,j} |,|P_{i-1,j}-P_{i,j} |,|P_{i,j+1}-P_{i,j} |,|P_{i,j-1}-P_{i,j} |>th$, where $th=0.002$. $Q_{i,j}=0$ elsewhere. th is the gradient of the second derivative at the point where the second derivative is zero, i.e. on the edge. Particles are found by searching for closed contours in this binary image.

\subsection{Comparing Transparent Microspheres and Dots}

A spherical lens scatters almost all the incident light in different directions, leaving only a bright spot in the middle where light is transferred directly through. For larger particles in focus, this bright spot will result in a second circular closed contour inside the outer edge. 

Small water droplets can be seen as transparent microspheres. The composition changes the refractive index, but this has little effect on the shadow. The outer contour is the same following the same reasoning as with the dots used for calibration \cite{ryd2015}. Diffraction patterns depend on the wavelength of the light and the size of the sphere. The resolution of the constructed system is not high enough for these patterns to be visible.

By using a flood-fill function on the binary image containing the detected contour, starting in a point at a minimum distance from an object, the edge of the filled area will border to the outer contour. This makes it possible to select the outer closed contours of possible particles.

\subsection{Removing the Center Bright Spot}

The center bright spot will make the shadow of a transparent sphere brighter in average than a solid dot. This will have an impact on the size calculation, since the size calculation is based on the shadow impact and the calibration is done using solid dots. Therefore we apply a mask on the image before calculating the size of the shadow. The mask is done by replacing all the centermost pixels in a shape of a circular disc with the intensity of the darkest pixel in the spot. The diameter of the masked disc is the arc length of the edge contour divided by 4$\pi$. The bright spot is measured by calculating the difference between the least value and the center pixel. Only particles with a difference larger than 0.1 (ten percent) will have the mask applied

\subsection{Measuring Roundness}

There may be clogs of small microspheres or dust in the samples that are measured. Microspheres, or small water droplets are close to spherical. Therefore we try to exclude all objects that the program find but are not circular. After the detection of object edges, we measure the roundness of each object. A measure of the mean square roundness deviation similar to the one described by ISO \cite{iso12181} can be achieved by calculating the quote between the area of the contour and the square of the total arc length See (\cref{eq:5}).
\begin{equation}
roundness = 4\pi \frac{A_{contour}}{arcLength^2}
\label{eq:5}
\end{equation}
$A_{contour}$ is the pixel area of the contour and $arcLength$ is the perimeter of the measured closed contour. In the calibration and field measurements described, an object is only considered spherical if $roundness \geq 0.85$.


\section{The Fog Chamber}
The fog chamber is needed to create a test environment for the instrument. Natural fogs tend not to occur outside just when we are ready to test. If possible, it is desirable that the environmet can be controlled and verified using a second instrument. It gives an indication of how the instrument will behave in a real measurement. It is also a verification of the instrument’s water ingress resistance.

The constructed chamber has a frame made of 30 mm aluminum profiles, fitted with transparent 6 mm polycarbonate walls on all sides using rubber sealing strips. The droplets are produced using an ultrasonic fog generator pushing the droplets to the chamber through a flexible tube approximately 30 mm in diameter and 500 mm long. Next to the fog inlet, there is a dry air inlet with a speed adjustable fan. On the back of the chamber there is a similar sized outlet for air and moisture.
 
\begin{figure}[ht]
\centering\includegraphics[width=0.75\linewidth]{figures/DSC_0103}
\caption{Fog chamber with connected droplet generator (blue container) and a multimeter used for fan power measurement. A Beaglebone Black microcontroller (blue box) is used for fan speed regulation.}
\end{figure}

\section{The Klövsjö Installation}
Figure \ref{fig:installation1} shows the installation. On top is the two camera houses of the DII and just below is the smaller CDP. The Lambrecht Eolos weather sensor is seen furthest to the left, mounted on an horizontal boom. In the middle just right of the Eolos is the mobile communication antenna. In the lower center the top of the shortened lattice mast is seen and behind this is the box containing the DII processing computer, the CDP data collection computer and the communications router. The whole installation is about five meters high. An electric servomotor mounted at the base of the pole inside the lattice mast rotates the two instruments automatically to follow the horizontal direction of the wind. 

\Cref{fig:installation3} shows icing on the front side of the installation.

\begin{figure}[ht]
\centering\includegraphics[width=0.75\linewidth]{figures/installation3}
\caption{Icing on the front side of the installed instruments.}
\label{fig:installation3}
\end{figure}

\begin{figure}[ht]
\centering\includegraphics[width=0.75\linewidth]{figures/installation1}
\caption{Installation in Klövsjö.}
\label{fig:installation1}
\end{figure}

% this is shows the paper size and measures
%\begin{figure}
%    \oddpagelayouttrue
%    \twocolumnlayoutfalse
%    \stockdiagram
%    \caption{Right-hand page major layout parameters for
%        the \file{memoir} class} \label{fig:mempplt}
%\end{figure}
%\stockvalues
%
% !TeX root = ../main.tex
% !TeX spelling = en_GB
% !TeX program = pdflatex

\chapter{Discussion}
\label{chap:discussion}

\section{Starting Studies}
Infrared and Visible Light

Laser Light

Image Noise

Light Sensitivity Measurement
\subsection{Speed of Light Flash}

\subsection{Aerodynamics}
Many existing instruments suffer from errors caused by the instrument itself during sampling, e.g. when droplets get stuck on the inlet (Spiegel et al., 2012), or shatters into smaller droplets (Cohen, 1991).
At higher wind speeds, there may be an effect of particles shattering and bouncing on the supporting structure making up the instrument (Field, Heymsfield, \& Bansemer, 2006).
An instrument should be designed in order to affect the free flow of particles as little as possible (D. Baumgardner et al., 2011).


\subsection{Optics}
Using a high power LED instead of laser reduces the interference effects used in e.g. holography (Henneberger et al., 2013), but since it is a monochromatic source, interference may not be completely ruled out. The coherence length, provided a Gaussian spectrum, can be approximated by Eq. 12 (Akcay, Parrein, \& Rolland, 2002): $equation12λ$

LEDs can sometimes be used with currents far above the specifications, as long as the pulse length is short and the duty cycle is low enough to permit the heat generated to be transported away between the pulses. Using the LED above specifications may though affect the efficiency and aging of the LED. LED emittance also depends on the temperature. Depending on the capacitance of the diode, the rise time may limit the current, although there exist some techniques to shorten the LED pulses (Tanaka, Umeda, \& Takyu, 2011; Veledar et al., 2007).



\todo{check these facts}


\section{Droplets and Ice Interference}

Droplets that are very close are likely to coalesce, thereby decreasing the number concentration at a rate that appears to increase for larger droplets and more complex droplet size distributions (Bordás, Hagemeier, Wunderlich, \& Thévenin, 2011). 

Due to the small depth of the measurement volume, i.e. the measuring range compared with the field of view, the likelihood of finding two droplets very close in the image is very low due to the low number concentration of droplets we are measuring.

A solution is to make a measurement of the droplet’s circularity and add this as selection criteria for the measurement. This solution also works as a filter for ice or snow particles. 

\section{Icing}

the risk of icing increases with increasing wind speed (Lasse Makkonen, 2000)



Calibration of true droplet size and measuring range both depends on the measured size of the droplet shadow and the amount of light used for exposure. It is possible to predict both the precision of the measured size and accuracy for measurement of droplet size. 

A value of both MVD and LWC can be derived from a series of images and since the number of measured droplets will depend on the concentration, the accuracy and precision will depend on the number of samples from the total population of droplets. 

Many existing instruments suffer from errors caused by the instrument itself during sampling, e.g. when droplets get stuck on the inlet [19], or shatters into smaller droplets [20]. An instrument should be designed in order to affect the free flow of particles as little as possible [16]. Therefore we also investigate the illuminative power required to get a good exposure with the tested system at a targeted maximum wind speed of 50 m/s.

We used a stage micrometer scale for characterization of the system and simulation of water droplets. This characterization holds true given that the optical silhouette of a droplet is comparable to a dot having equal diameter and being printed on a silicon glass. It is not a new concept and has at least once been proved experimentally for coherent light, by comparing with beads of glass and water droplets of known sizes [21]. The shadow image of water drops of any size will be defined mainly by the diffracted component, as long as the distance between the drop and the lens is much larger than the drop diameter. Only in a small bright spot in the middle will the refracted component be large enough to be visible. [21, 22]. 
Using the results of this study, a weather protected prototype may be built to perform a comparative study.
We believe that this study of a shadowgraph imaging system provides good analysis of its expected major limitations related to the measurement of liquid water content of air. This is also the scientific contribution of this publication. 


The design using a weakly collimated LED that illuminates an area slightly larger than the field of view makes the system quite insensitive to misalignment of the camera and the light source. Temporal or permanent changes in light intensity caused by a minor misalignment can be automatically compensated for by continuous measurement of the total exposure level. If the level of exposure is increasing or decreasing, the length of the light pulse is changed correspondingly. The light intensity can also be affected by dirt on the front glass of the housings. 

Spatial dissimilarities in the light intensity that are not caused by noise we can compensate for by calculating the local average intensity of the background around each measured droplet. The size of a droplet is then based on the intensity dip caused by the shadow compared with its local background.


making a flat-field correction. This is done by constructing an average image by the last 20 images and using this as a 





%
\chapter{Conclusion and Outlook}
\label{chap:conclusion_outlook}

\section{Conclusions}
The work shows that shadowgraph imaging can be used for precise measurements of the LWC and the MVD. Edge detection can be used to find droplets in an image and the measurement volume can be defined from a measure of the edge gradient.

The used LED is capable of emitting at much higher currents than the specified max.

The DII was proven to withstand and function in a very humid environment. 

The LWC in a fog chamber can be controlled by regulating the fan speed and the power of an ultrasonic humidifier. 


The resulting LWC and MVD measured by the DII can be compared with the CDP using a 30 minute average. The DII measures a lower LWC 

\section{Future Work}

\begin{enumerate}
\item
A second study of real world LWC and MVD should be done using a third instrument. 
\item
A simultaneous measurement of ice load should be done to find out more about the relationship between MVD, LWC and ice load.
\item
The DII should be improved by increasing the image processing speed. This can be done e.g. by implementing pre-processing algorithms in hardware or by switching to a more powerful processing computer.
\item
Higher wind speeds can e used. The LED flash can be shortened by using higher. This means changing the hardware that drives the LED to achieve a higher current.
\end{enumerate}


A study to further investigate the relation between the different parameters impact on the ice load may be done using the instruments an ice monitoring device. This should be done in combination with a third independent LWC and MVD measurement, e.g. by using rotating cylinders\cite{makk1992}.

\section{Contributions}
Paper I
Paper II
Paper III
Paper IV

%
% -- start the final content -------------------------------
%\backmatter

%\glsaddall
%\printglossary[style=list, type=\acronymtype, title=Acronyms]

\printbibliography[heading=bibintoc]
\bibintoc

%\input{./backmatter/s}


%% !TeX root = ../main.tex
\cleardoublepage
\pagestyle{plain}
\refstepcounter{dummy}
\label{pap:paper1}
{%\tikzexternaldisable 
    \begin{tikzpicture}[overlay,remember picture]
    \node[fill=black, rectangle, text width=4cm,
    text height=1.5em,anchor=north east, inner sep=12pt] 
    at ($ (current page.north east) + (-0cm,-3cm) $) 
    {\textcolor{white}{\HUGE Paper I}};
\end{tikzpicture}}
\vfill
\noindent
\LARGE \paperone
\vspace{0.5cm}

\normalsize\noindent

\authorone
% or authors command
\vspace{2cm}

\cleardoublepage % the shim needs to appear on a righthandside page and the lefthandside page immediately following needs to be blank

\includepdf[pages=1-,width=1.0\textwidth,pagecommand=\makeoddfoot{miunlic}{}{}{\small page | \thepage}, templatesize={169mm}{239mm}, frame=false, scale=1, trim=29mm 5mm 29mm 10mm, offset=-5 0]{papers/Paper1.pdf} %note: I don't know why \makeoddfoot creates page numbers for both even and odd pages, but it seems to work
 %note: My solution for the shims is not elegant, but it got the job done. Not sure this is the best solution for the template though. WW
%\input{./papers/paper2}
%\input{./papers/paper3}
%\input{./papers/paper4}

\end{document}

% Alternate way of implementing bibliography (Added 02/05/12 by WW)
%comment out previous four lines and use instead;

%\bibliographystyle{ieeetr} %these three lines can be placed here
%\renewcommand \bibname{References}
%\bibliography{./10_backmatter/bibliography}

%For help on syntax in the .bib file, go to http://en.wikibooks.org/wiki/LaTeX/Bibliography_Management

%Mendeley can be used to manage bib files!
% there is a bash script to update these from Mendeley

% //////////////////////// THE END \\\\\\\\\\\\\\\\\\\\\\\\ 
