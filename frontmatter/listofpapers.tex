% !TeX root = ../main.tex

%\chapter{List of Papers}

\thispagestyle{plain}

\chapter*{List of Papers}
%\noindent {\Huge\bfseries\sffamily List of Papers}\\
\refstepcounter{dummy}
\addcontentsline{toc}{chapter}{List of Papers}
\vspace{20pt}

\noindent This thesis is based on the following papers, herein referred to by their Roman numerals:  

% paper name in a short command
\newcommand{\paperone}{Liquid Water Content and Droplet Sizing Shadowgraph Measuring System for Wind Turbine Icing Detection}
\newcommand{\papertwo}{Droplet Imaging Instrument - Metrology Instrument for Icing Condition Detection}
\newcommand{\paperthree}{Size measurement validation of the droplet imaging instrument}
\newcommand{\paperfour}{Field Study of Ground-LWC using the Droplet Imaging Instrument}

% authors in a short command
\newcommand{\authorone}{S.Rydblom, B. Thörnberg}
\newcommand{\authortwo}{S.Rydblom, B. Thörnberg, P. Jonsson, E. Olsson}


%some journals in short
\newcommand{\ieeesens}{IEEE Sensors Journal}
\newcommand{\ieeeconf}{IEEE International Conference on Imaging Systems and Techniques (IST) Proceedings}
\newcommand{\mst}{Measurement Science and Technology}
\newcommand{\coldreg}{Cold Regions Science and Technology}

\begin{description}[style=nextline]
    \item[Paper I]
    \paperone \\ 
    \authorone, \ieeesens, 2015\dotfill \pageref{pap:paper1}
    
    \item[Paper II]
    \papertwo \\ 
    \authorone, \ieeeconf, 2016\dotfill \pageref{pap:paper2}
 
    \item[Paper III]
    \paperone \\ 
    \authorone, \mst, 2017\dotfill \pageref{pap:paper3}
    
    \item[Paper IV]
    \paperfour \\ 
    \authortwo, \coldreg, 2017\dotfill \pageref{pap:paper4}

%    \item[Paper V]
%    
%    \item[Paper VI]
%   
%    \item[Paper VII]
%
%    \item[Paper VIII]



\end{description}

