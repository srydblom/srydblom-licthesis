% !TeX root = ../main.tex
% !TeX spellcheck = en_SV
%: Abstract swedish

\begin{abstract}

Den här avhandlingen beskriver utforskandet av en metod att mäta storlek och koncentration av vattendroppar i atmosfären. Avsikten är att hitta en kostnadseffektiv teknik för att förutsäga isbildning.

Isbildning som orsakas av atmosfäriskt vatten kan vara ett signifikant problem för vår infrastruktur som t.ex. kraftledningar, vägar och flygtrafik. Omkring en tredjedel av den globala installerade vindkraften anses vara i kalla klimat, där nedisning av rotorbladen är en av de största utmaningarna.

Nedisningsprocessen är komplex och resultatet beror på en kombination av strukturen eller vingens aerodynamiska form, den förbipasserande luftens hastighet, luftens och ytans temperatur, eventuell blandning av snö och vatten, concentrationen av flytande vatten och dessa vattendroppars storleksdistribution.

Den valda mätmetoden baseras på skuggfotografering med en LED som bakgrundsbelysning samt digital bildbehandling. Ett prototypinstrument är konstruerat med övervägande standardkomponenter. Dessa har valts med tanke på möjligheten till låg produktionskostnad. Exempel på applikationer för ett kommersiellt instrument baserat på denna teknik är villkorsmätning för nedisning i realtid, assimilation och verifikation av data i numeriska vädermodeller.

Arbetet som presenteras visar att mätning av storlek och koncentration av vattendroppar med hjälp av skuggfotografering kan användas för validering av NWP-modeller och andra instrument. Noggrannheten hos storleksmätningen är hög. Noggrannheten av koncentrationsmätningen har potential att bli hög eftersom varje droppe får en individuellt kalibrerad mätvolym. Precisionen hos mätinstrumentet beror till största delen på hur många bilder och vattendroppar som används för att skapa varje mätvärde. Instrumentets prestanda för att mäta i realtid begränsas av kamerans bildhastighet och bildbehandlingstiden samt vilken precision i mätvärdet som önskas.

\end{abstract}