% !TeX root = ../main.tex
% !TeX spellcheck = en_GB
%: Abstract english
\begin{abstract}

This thesis describes the exploration of a method to measure the droplet size and the concentration of atmospheric liquid water. The purpose is to find a cost efficient technique to detect the conditions for icing on structures.

Icing caused by freezing atmospheric water can be a significant problem for infrastructure such as power lines, roads and air traffic. About one third of the global installed wind power capacity is located in cold climates, where icing of rotor blades is one of the major challenges. 

The icing process is complex and the result depends on a combination of the aerodynamic shape of the structure or airfoil, the velocity of the air and its contained water, the temperature, the mixing of snow and water, the concentration of liquid water and the droplet size distribution.

The chosen measurement method is based on a shadowgraph imaging system using \gls{led} light as background illumination and digital image processing. A prototype instrument has been constructed. The components were selected keeping the possibility of low-cost volume production in mind. The applications of a commercial instrument based on this technique are e.g. real-time in-situ icing condition measurements and assimilation and verification of data in numerical weather models.

The work presented shows that measurements of the size and concentration of water droplets using shadowgraph images can be used for the comparison and validation of \gls{nwp} models and other instruments. The accuracy of the particle size measurement is high. The accuracy of the concentration measurement has the potential to become high due to the single-particle measurement range calibraiton. The precision of the instrument depends mainly on the number of images that is used to find each measurement value. The real-time performance of the instrument is limited by the image retrieval and processing speed and depends on the precision required.

\end{abstract}