\chapter{Conclusion and Outlook}
\label{chap:conclusion_outlook}

\section{Conclusions}
The work shows that shadowgraph imaging can be used for precise measurements of the LWC and the MVD. Edge detection can be used to find droplets in an image and the measurement volume can be defined from a measure of the edge gradient.

The used LED is capable of emitting at much higher currents than the specified max.

The DII was proven to withstand and function in a very humid environment. 

The LWC in a fog chamber can be controlled by regulating the fan speed and the power of an ultrasonic humidifier. 


The resulting LWC and MVD measured by the DII can be compared with the CDP using a 30 minute average. The DII measures a lower LWC 

\section{Future Work}

\begin{enumerate}
\item
A second study of real world LWC and MVD should be done using a third instrument. 
\item
A simultaneous measurement of ice load should be done to find out more about the relationship between MVD, LWC and ice load.
\item
The DII should be improved by increasing the image processing speed. This can be done e.g. by implementing pre-processing algorithms in hardware or by switching to a more powerful processing computer.
\item
Higher wind speeds can e used. The LED flash can be shortened by using higher. This means changing the hardware that drives the LED to achieve a higher current.
\end{enumerate}


A study to further investigate the relation between the different parameters impact on the ice load may be done using the instruments an ice monitoring device. This should be done in combination with a third independent LWC and MVD measurement, e.g. by using rotating cylinders\cite{makk1992}.

\section{Contributions}
Paper I
Paper II
Paper III
Paper IV
