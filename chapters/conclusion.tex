\chapter{Summary of the Thesis}
\label{chap:summary}

The thesis describes the work resulting in three scientific papers.

\section{General Conclusions}

Shadowgraph imaging can be used for comparison and validation of NWP models.It can also be used as reference measurement for instruments using other techniques. The accuracy of the particle size measurement is high. 

The accuracy of the concentration measurement is questioned due to the systematic difference to the reference instrument, but has the potential to become high due to the single-particle measurement range calibraiton. The precision of the instrument depends mainly on the number of images that is used to achieve each measurement value. 

The real-time performance of the instrument is limited by the image retrieval and processing speed and depends on the precision required.

\section{Other Conclusions}

\begin{enumerate}
\item
The test using polymer microspheres confirms that the DII, calibrated by a micrometer dot scale measures the size of the tested samples correctly. The size distributions achieved by this instrument are within the tolerances specified by the manufacturer of the microspheres.

\item
The used LED is capable of emitting at much higher currents than the specified maximum.

\item
The DII was proven to withstand and function in a very humid environment. 

\item
The LWC in a fog chamber can be controlled by regulating the fan speed and the power of an ultrasonic humidifier. 

\item
Both the CDP and the DII make precise measurements of the LWC using a 30 minute average window. Supercooled droplets with diameters above 50 µm  exist in fogs where MVD is lower than predicted.

\item
The CDP achieves higher values of the droplet diameters compared with the DII. This leads to an even higher difference in the measured LWC.

\item
A calibration of the measurement range depending on the distance from the measured particle to the optical center is needed in order to get a better estimation of the measurement volume.

\item
The predicted LWC and MVD data from HARMONIE-AROME have better agreement with the measured values when using a 500 meter horizontal resolution than the usual 2.5 km resolution.

\item
The DII proved to be fully operational without site attendance for four months of continuous measurement. The instrument speed and resolution seems to be good enough to detect and measure the conditions for icing. The data can be used to verify and validate NWP models.

\item
Large droplets are important to understand the total size distribution of liquid water droplets and may play an important role during icing. 

\end{enumerate}

\section{Future Work}

\begin{enumerate}
\item
A second study of real world LWC and MVD should be done using a third instrument. 

\item
A simultaneous measurement of ice load should be done to find out more about the relationship between MVD, LWC and ice load.

\item
The DII should be improved by increasing the image processing speed. This can be done e.g. by implementing pre-processing algorithms in hardware or by switching to a more powerful processing computer.

\item
Higher wind speeds can be used. The LED flash can be shortened by using higher. This means changing the hardware that drives the LED to achieve a higher current.

\item
A study to further investigate the relation between the different parameters impact on the ice load may be done using the instruments an ice monitoring device. This should be done in combination with a third independent LWC and MVD measurement, e.g. by using rotating cylinders\cite{makk1992}.

\end{enumerate}
