% !TeX root = ../main.tex
% !TeX spelling = en_GB
% !TeX program = pdflatex

\chapter{Discussion}
\label{chap:discussion}

\section{Starting Studies}

\subsection{Infrared and Visible Light}

\subsection{Laser Light}

\subsection{Image Noise - SNR}


Light Sensitivity Measurement
\subsection{Speed of Light Flash}

The LED used was fully functional after four months of continous measurements using 12 times higher current than specified. Using even higher current 

\section{Aerodynamics}

\section{Difficulties with Polymer Microspheres}
To get a second verification of the sizing algorithm we made a measurement using polymer microspheres of known calibrated size distributions.
\subsection{Speed}


\subsection{Clogging}


\section{Optics}
Using a high power LED instead of laser reduces the interference effects used in e.g. holography \cite{henn2013}, but since it is a monochromatic source, interference may not be completely ruled out. The coherence length, provided a Gaussian spectrum, can be approximated by Eq. 12 \cite{akcay2002}: $equation12λ$

LEDs can sometimes be used with currents far above the specifications, as long as the pulse length is short and the duty cycle is low enough to permit the heat generated to be transported away between the pulses. Using the LED above specifications may though affect the efficiency and aging of the LED. LED emittance also depends on the temperature. Depending on the capacitance of the diode, the rise time may limit the current, although there exist some techniques to shorten the LED pulses \cite{tanaka2011,vele2007}.

The LED used was fully functional after four months of continous measurements using 12 times higher current than specified. At 12 ampere driving current, the flash duration could be lowered to approximately 250 ns, still using the normal settings of amplification in the camera. Using even higher current would lower the flash duration still.


\section{Droplets and Ice Interference}

Droplets that are very close are likely to coalesce, thereby decreasing the number concentration at a rate that appears to increase for larger droplets and more complex droplet size distributions (Bordás, Hagemeier, Wunderlich, \& Thévenin, 2011). 

Due to the small depth of the measurement volume, i.e. the measuring range compared with the field of view, the likelihood of finding two droplets very close in the image is very low due to the low number concentration of droplets we are measuring.

A solution is to make a measurement of the droplet’s circularity and add this as selection criteria for the measurement. This solution also works as a filter for ice or snow particles. 

\section{Icing}

the risk of icing increases with increasing wind speed (Lasse Makkonen, 2000)



Calibration of true droplet size and measuring range both depends on the measured size of the droplet shadow and the amount of light used for exposure. It is possible to predict both the precision of the measured size and accuracy for measurement of droplet size. 

A value of both MVD and LWC can be derived from a series of images and since the number of measured droplets will depend on the concentration, the accuracy and precision will depend on the number of samples from the total population of droplets. 

Many existing instruments suffer from errors caused by the instrument itself during sampling, e.g. when droplets get stuck on the inlet [19], or shatters into smaller droplets [20]. An instrument should be designed in order to affect the free flow of particles as little as possible [16]. Therefore we also investigate the illuminative power required to get a good exposure with the tested system at a targeted maximum wind speed of 50 m/s.

We used a stage micrometer scale for characterization of the system and simulation of water droplets. This characterization holds true given that the optical silhouette of a droplet is comparable to a dot having equal diameter and being printed on a silicon glass. It is not a new concept and has at least once been proved experimentally for coherent light, by comparing with beads of glass and water droplets of known sizes [21]. The shadow image of water drops of any size will be defined mainly by the diffracted component, as long as the distance between the drop and the lens is much larger than the drop diameter. Only in a small bright spot in the middle will the refracted component be large enough to be visible. [21, 22]. 
Using the results of this study, a weather protected prototype may be built to perform a comparative study.
We believe that this study of a shadowgraph imaging system provides good analysis of its expected major limitations related to the measurement of liquid water content of air. This is also the scientific contribution of this publication. 


The design using a weakly collimated LED that illuminates an area slightly larger than the field of view makes the system quite insensitive to misalignment of the camera and the light source. Temporal or permanent changes in light intensity caused by a minor misalignment can be automatically compensated for by continuous measurement of the total exposure level. If the level of exposure is increasing or decreasing, the length of the light pulse is changed correspondingly. The light intensity can also be affected by dirt on the front glass of the housings. 

Spatial dissimilarities in the light intensity that are not caused by noise we can compensate for by calculating the local average intensity of the background around each measured droplet. The size of a droplet is then based on the intensity dip caused by the shadow compared with its local background.


making a flat-field correction. This is done by constructing an average image by the last 20 images and using this as a 




