% !TeX root = ../main.tex
% !TeX spelling = EN_gb


\chapter{Introduction}
Measuring and controlling the properties of a fog of water droplets has been the interest and focus of many studies during at least half a century. Applications range from atmospheric studies, aircraft safety to military and commercial applications. 

Icing caused by freezing atmospheric water can be a significant problem in cold climates, affecting infrastructure such as wind turbines, power lines, and road and air traffic. With the increasing importance of electric power generated by wind, there is a renewed demand predictions of icing on wind turbines. Icing on the blades of a turbine lowers the efficiency, increases noise and may force the turbine to a complete stop [1-4]. Aircraft, power lines or any other weather exposed structure share this problem. Therefore big efforts have been made to create models for how the ice is formed [5, 6] and how it can be included in weather prediction models [7, 8].

This thesis is an exploration of a cost effective and robust method to measure atmospheric liquid water in order to predict icing on ground based structures.

The measurement method is based on digital image processing in a shadowgraph system using LED light as background illumination. A prototype instrument is built and initially tested using a fog chamber. It is calibrated using a micrometer dot scale with circular discs printed on a silicon glass. Polymer microspheres are also used for size measurement verification.

The instrument is constructed using standard components with the intent of viewing possibilities of low cost volume production. It may e.g. be used for real-time icing condition measurement, or remote meteorological data collection. 

Following is a brief description of related work and an introduction to the attached papers.

\section{Related Work}

In 1970, Knollenberg [23] described an electro-optical technique to measure cloud and precipitation particles using a laser illuminated linear array of photo detectors. The photo detectors are used to make a two-dimensional image of the particles’ shadows as they pass the light beam. Systems based on this technique are called Optical Array Probes (OAP) or two-dimensional imaging probes. Later development of this technique includes using image sensors to save gray scale images of the detected particles [17, 22].

Light scattered from a focused laser  [15, 24] is one common technique to measure single particles. A laser beam is used to illuminate passing particles. When a particle is detected, its size is determined by comparing the variations in light with a pattern derived from the Mie scattering solutions [25].

The OAPs and the light scattering spectrometers each have some advantages over the other technique depending on the nature of the aerosol. Instruments for airborne use have been developed that combine several techniques into one single probe [26] for accurate measuring of LWC and MVD [16].

A similar but different optical technique for measuring water droplets is based on in line holography [27]. In principle this is a two-dimensional shadowgraph imaging system that use laser background illumination to create images of the diffraction patterns created by the passing particles. These patterns are measured to reconstruct images of the particles. This is a fairly calculation intensive process, but which may be one of the reasons why instruments based on this technique are not so common [28].

Sizing of the droplets using Mie spectrometry is a complex operation even for coherent light [29], and although it is possible to study Mie scattering from white light [30], the complexity, small droplet size range and sample volume makes it less attractive in this application. The optical resolution is usually too low. Therefore it is also difficult to determine the exact particle size and the usable depth of field for incoherent shadowgraph systems. The shadow from a particle can e.g. appear smaller or larger when out of focus.

Shadowgraph imaging of particles using incoherent illumination instead of laser has been tried e.g. in particle shadow velocimetry (PSV) [31, 32], or spray characterization [33]. Quantitative and comprehensive studies of other droplet measurement techniques exist [24, 28, 34].

Kuhn et al (2012) described a method to characterize ice particles by Fourier shape descriptors (Granlund 1972). The system uses a microscope-like technique to achieve a high resolution level, in the order of one micrometer. Perhaps this method can be used also for larger scale objects like snowflakes. The system described by Kuhn et al has a field of view of 200x150um and a depth of field of approximately ten micrometers.



