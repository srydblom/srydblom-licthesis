% !TeX root = ../main.tex
% !TeX spelling = EN_gb


\chapter{Introduction}

This thesis describes the exploration of a method to measure the size and concentration of atmospheric liquid water, with the purpose of finding a cost efficient technique to detect the conditions for icing on structures. The chosen measurement method is based on digital image processing and a shadowgraph imaging system using \gls{led} light as background illumination. A prototype instrument is manufactured using components with the intent of viewing possibilities of low cost volume production. The applications of a commercial instrument based on this technique are e.g. real-time in-situ icing condition measurements and assimilation and verification of data in numerical weather models. 

Paper I describes a shadowgraph system that uses edge detection to define a measurement range depending on the size of a particle and the image signal to noise ratio (\gls{snr}). Paper II describes the size measurement method, a comparison with  physical design of the instrument integrated with weather protection a fog chamber used for climate tests. Paper III describes a verification of the size measurement using polymer microspheres in calibrated (\gls{nist} traceable) distributions and a field study of ground \gls{lwc} and \gls{mvd} using the \gls{dii} and a comparative study using the Cloud Droplet Probe (\gls{cdp}).

\section{Atmospheric Water and Icing}

Measuring or controlling the properties of a fog of water droplets has been the interest and focus of many studies during at least half a century. Applications can be found in many different fields, like atmospheric studies, aircraft and road traffic safety, medicin, infrastructure and military defence. When a mass of air is saturated with water vapour the abundant water will be in either liquid or solid form. In order to for water to freeze, or change from liquid to solid state, it needs to be below zero degrees Celsius at the standard atmospheric pressure (273.15 K at 101.325 kPa). For an ice crystal to form in the atmosphere, unless the temperature is below -40°C, the presence of an ice nucleus is also required. If no ice nucleus is present, the water droplet will remain in the liquid but supercooled condition. In this condition the water freeze very quickly on the contact with a cold surface. Depending on the size of the droplets, the icing is called in-cloud icing, freezing rain or wet snow icing. Combinations of these weather conditions are common.

Icing caused by freezing atmospheric water can be a significant problem in cold climates, affecting infrastructure such as wind turbines, power lines, road and air traffic. With the increasing importance of electric power generated by wind, there is a demand for predictions of icing on wind turbines. Icing on the blades of a turbine lowers the efficiency, increases noise and may force the turbine to a complete stop \cite{dalili2009,cost727,homo2012,jasin1998}. Aircraft, power lines or any other weather exposed structure share this problem. Efforts have been made to create models for how the ice is formed \cite{makk2000,makk2001,shin1992} and how it can be included in weather prediction models \cite{thomp2009,kring2011}. The severity of the icing depends on many factors, e.g. the duration, the shape and surface of the structure, the wind speed, the atmospheric concentration of liquid water, the size distribution of water droplets and the mix of ice and snow \cite{kring2011,makk2000,makk2001,homo2010,han2012}. Of particular interest for the prediction of effect loss on a wind turbine is the phase of ice accretion \cite{davis2014}.

The Liquid Water Content (\gls{lwc}) in atmospheric studies is usually given in either $\mathrm{g\cdot m^{-3}}$ \cite{sein1998} or as the mass quote of water and air $\mathrm{g\cdot kg^{-1}}$. The size distribution of droplets when considering the collision efficiency is most effectively estimated by the Median Volume Diameter (\gls{mvd}) \cite{fins1988}. The \gls{mvd} is the point in the distribution of droplet diameters where half of the total amount of water in the distribution of droplets is above the diameter, while the other half is below. Weather models today include the \gls{lwc} and the \gls{mvd}, making it possible to estimate the risk of icing based on general weather data \cite{thomp2009}. The \gls{lwc} and the \gls{mvd} at ground level depends very much on the local geographical topology, and measurements using in situ instruments are quite rare. There have been trials to verify NWP models using in-situ data \cite{berg2013}. The \gls{lwc} and the \gls{mvd} can e.g. be measured by analyzing satellite or radar images. In-situ measurements are rarely simple or straightforward, especially considering the occurance of mixed conditions mentioned. Despite the development of diverse measurement techniques and instruments, in-situ measurements are still quite rare due to the cost and that most instruments require more or less installation expertize, maintenance and periodic calibration.

\section{Justification and Narrative}

Regions where icing events or periods with temperatures below the standard operational limits occur are of great interest for the installation of wind power. About one third of the global installed wind power capacity is considered to be in these cold climates, where icing of rotor blades is one of the major challenges \cite{iea2017}.

The work leading to this thesis started with an open mindset and the goal set to find a method to detect icing on wind turbines by measuring the optical properties of liquid water in clouds. We knew from previous projects how the reflectance and absorbtion of light changes depends on the wavelength. The absorption of light is comparatively high for some wavelengths in near and far infrared, but low in the visible blue and green. Clouds consist of water droplets, but the absorption if a function of the travelled distance, which makes it very small in small droplets. Most of the light will be scattered, which is why clouds appear gray at a distance. Light with strong coherence, like that from lasers, will be affected by interference. Coherent and incoherent light sources in wavelengths from visible (450 nm) to near infrared (850 nm) were tested. Illumination angles were tested from 90 (sideways) to 180 (backlit) degrees. After evaluating the different techniques, 180 degrees backlit illumination using blue \gls{led} was selected.

Optical measurement of water droplets have been researched and resulted in many publications during the last decades. Shadowgraphy is also used in particle velocimetry, for analyzing sprays in many different applications. Therefore there are many publications and litterature available on these subjects. Still, there is no instrument readily available today that is affordable, reliable and possible to operate near the highest point where icing usually occur, like a hundred meters up in a wind turbine, or mast. An instrument for measuring icing parameters for wind turbines should both be able to detect supercooled cloud droplets from five to a few hundred micrometers in diameter and be able to determine the concentration to get an accurate measure of the \gls{lwc}.

The Droplet Imaging Instrument (\gls{dii}) was designed to find a simple, cost effective and robust technique to measure \gls{lwc} and \gls{mvd} in order to predict icing on structures. An automatic sensor based on this technique could be used to trigger ice protection systems used for wind turbines in cold climates \cite{lamra2013}. It was calibrated using a micrometer dot scale with dots of verified sizes printed on a transparent glass. The sizing is verified using four samples of polymer microspheres, calibrated in turn by the National Institute of Standards and Technology (\gls{nist}). The functional simplicity and robustness of the \gls{dii} opens the possibility to mass produce and distribute Internet connected instruments based on this technique on a larger geographical scale.

During the winter of 2016-2017, the prototype instrument was placed on the Klövjö mountain, Sweden, at one of the Swedish Meteorological and Hydrological Institute's (\gls{smhi}) measuring stations, position 62°29'41''N, 14°9'27''E, 802 meters above sea level. Measurements are available from November 2016 to March 2017. Results from this measurement together with a parallel measurement of \gls{lwc} and \gls{mvd} using a Cloud Droplet Probe \gls{cdp}-2 (\gls{cdp}) from Droplet Measurement Technologies Inc. (\gls{dmt}) are presented in Paper III as well as in this thesis.

\section{Related Work of Measurement}
\label{sec:relwork}
In 1970, Knollenberg \cite{knoll1970} described an electro-optical technique to measure cloud and precipitation particles using a laser illuminated linear array of photo detectors. The photo detectors are used to make a two-dimensional image of the particles’ shadows as they pass the light beam. Systems based on this technique are called Optical Array Probes (\gls{oap}) or two-dimensional imaging probes. Developments of this technique include using image sensors to save gray scale images of the detected particles \cite{kulk2011,wend2013}. The Droplet Imaging Instrument (\gls{dii}) developed and described in this thesis is based on this technique.

Light scattered from a focused laser \cite{baum1983,dye1984} is a different technique to measure single particles. A laser beam is used to illuminate passing particles. When a particle is detected, its size is determined by comparing the variations in light with a pattern derived from the Mie scattering solutions \cite{mie1908}. The Cloud Droplet Probe (\gls{cdp}) used as a reference instrument is based on that technique.

The \gls{oap}, imaging and the light scattering spectrometers each have some advantages over the other technique depending on the nature of the aerosol. Instruments for airborne use have been developed that combine several techniques into one single probe for accurate measuring of \gls{lwc} and \gls{mvd} \cite{baum2001, baum2011}.

A similar but different optical technique for measuring water droplets is based on in line holography \cite{laws1995}. In principle this is a two-dimensional shadowgraph imaging system that use laser background illumination to create images of the diffraction patterns created by the passing particles. These patterns are measured to reconstruct images of the particles. This is a fairly calculation intensive process, which may be one of the reasons why instruments based on this technique are not so common yet \cite{henn2013}. An advantage with this technique is the increased measurement range, i.e. the useable depth in each image \cite{kaikk2014}.

Sizing of the droplets using Mie spectrometry is a complex operation even for coherent light cite{bohr2008}, and although it is possible to study Mie scattering from white light \cite{ward2008}, the complexity, small droplet size range and sample volume makes it less attractive in this application. The optical resolution is usually too low. Therefore it is also difficult to determine the exact particle size and the usable depth of field for incoherent shadowgraph systems. The shadow from a particle can e.g. appear smaller or larger when out of focus.

Shadowgraph imaging of particles using incoherent illumination instead of laser has been tried e.g. in particle shadow velocimetry (PSV) \cite{este2005}, or spray characterization \cite{will2010}. Quantitative and comprehensive studies of other droplet measurement techniques exist \cite{dye1984,henn2013,conno2007}.

Kuhn et. al. \cite{kuhn2012} described a method to characterize ice particles by Fourier shape descriptors \cite{gran1972,walla1980}. The system uses a microscope-like technique to achieve a high resolution level, in the order of one micrometer. Athough not specifically designed to measure water droplets, the principal design is similar to the \gls{dii}. 



